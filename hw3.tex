\documentclass[12pt]{report}

\usepackage{fullpage}
\usepackage{amsmath,amssymb,bm,upgreek,mathrsfs}
\usepackage{algorithmic,algorithm}
\usepackage{graphicx,subcaption}
\usepackage{setspace}
\usepackage{color}
\usepackage{multirow}
\usepackage{alltt}
\usepackage{cancel}
\usepackage{listings}

\doublespacing

\DeclareMathOperator*{\argmax}{arg\,max}
\DeclareMathOperator*{\argmin}{arg\,min}

\newcommand{\N}{\mathcal{N}} \newcommand{\U}{\mathcal{U}}
\newcommand{\Poi}{{\text Poisson}} \newcommand{\Exp}{{\text Exp}}
\newcommand{\G}{\mathcal{G}} \newcommand{\Ber}{{\text Bern}}
\newcommand{\Lap}{{\text Laplace}} \newcommand{\btheta}{\boldsymbol{\theta}}
\newcommand{\bSigma}{\boldsymbol{\Sigma}}

\newcommand{\E}[1]{\mathbb{E}[#1]}
\newcommand{\Cov}[2]{\mathbb{C}\mathrm{ov}(#1,#2)}

\def\*#1{\mathbf{#1}} \newcommand*{\V}[1]{\mathbf{#1}}

%%%%%%%%%%%%%%%%%%%%%%%%%%%%%%%%%%%%%%%%%%%%%%%%%%%%%%%%%%%%%%%%%%%%%%

\begin{document}

\centerline{\it CS 480 HW \#3}

\begin{enumerate}

\item[1.] Effort level.

\item[a.] The homework took me 12 hours.
\item[b.] N\textbackslash{A}
\item[c.] Used ChatGPT as a search tool to look up code syntax and
  documentation.

  \newpage

\item[2.] Principle component analysis using MATLAB.

\item[a.] Visualize images.

  Below in Figure 1, the first line of three.txt is shown.

  \begin{figure}[H]
    \centering \fbox{
      \begin{minipage}{.90\linewidth}
        \includegraphics[width=\linewidth]{code/q2a1.png}
        \caption{In MATLAB, visualization of the first line of three.txt is
          generated using the \texttt{imagesc()} function. The
          \texttt{imagesc()} function scales up the size of unconventional
          matrix slices that have to be displayed (such as this one line vector
          from three.txt) and maps the grayscale intensity values to colors
          based on a colormap for easier distinguishability. This produces the
          image as shown.}
        \label{fig:q2a1}
      \end{minipage}}
  \end{figure}

  \newpage

  Below in Figure 2, the first line of eight.txt is shown.

  \begin{figure}[H]
    \centering \fbox{
      \begin{minipage}{.90\linewidth}
        \includegraphics[width=\linewidth]{code/q2a2.png}
        \caption{Again in MATLAB, visualization of the first line of eight.txt
          is generated using the \texttt{imagesc()} function.}
        \label{fig:q2a1}
      \end{minipage}}
  \end{figure}

  \newpage

\item[b.] Sample mean.

  In Figure 3 below, \( \bar{X}\) is displayed as a 16$\times$16 grayscale
  image.

  \begin{figure}[H]
    \centering \fbox{
      \begin{minipage}{.90\linewidth}
        \includegraphics[width=\linewidth]{code/q2b.png}
        \caption{\( \bar{X}\) was computed in MATLAB and displayed as a
          16$\times$16 grayscale image using \texttt{imagesc()}.}
        \label{fig:q2a1}
      \end{minipage}}
  \end{figure}

  \newpage
\item[c.] Covariance submatrix.

  The sample covariance matrix is shown below as the output of the computation
  performed in MATLAB.

\begin{verbatim}
  59.167       142.15       28.682      -7.1786      -14.336
  142.15       878.94       374.14       24.128      -87.128
  28.682       374.14       1082.9       555.23       33.724
  -7.1786       24.128       555.23       1181.2       777.77
  -14.336      -87.128       33.724       777.77         1430
\end{verbatim}

\item[d.] Compute eigenvalues.

  The two largest eigenvalues computed in MATLAB are shown below.

\begin{verbatim}
237155.246290486
145188.352686825
\end{verbatim}
  \newpage
  In Figure 4 below, eigenvector \( v_1\) is shown.
  \begin{figure}[H]
    \centering \fbox{
      \begin{minipage}{.90\linewidth}
        \includegraphics[width=\linewidth]{code/q2c1.png}
        \caption{In MATLAB, eigenvector \( v_1\) is displayed using
          \texttt{imagesc()}.}
        \label{fig:q2a1}
      \end{minipage}}
  \end{figure}

  \newpage
  In Figure 5 below, eigenvector \( v_2\) is shown.
  \begin{figure}[H]
    \centering \fbox{
      \begin{minipage}{.90\linewidth}
        \includegraphics[width=\linewidth]{code/q2c2.png}
        \caption{In MATLAB, eigenvector \( v_2\) is displayed using
          \texttt{imagesc()}.}
        \label{fig:q2a1}
      \end{minipage}}
  \end{figure}

\item[e.] Matrix projection.

  Matrix multiplication was performed in MATLAB to get the projected coordinates
  of the first line of three.txt and eight.txt respectively as reported below.

\begin{verbatim}
  p3 =
  240.9       297.49
  p8 =
  -208      -594.72
\end{verbatim}

  \newpage

\item[f.] Average reconstruction error.

  The average reconstruction error was computed in MATLAB as shown below. The
  code is pasted to show my work, and the result is pasted as the last line.

\begin{verbatim}
A = V*V';
err = [];
for i = 1:size(X,1)
    err = [err; X(i,:)*A - X(i,:)];
end
size(err);
s = sum(err.^2,2);
size(s);
format long g;
disp(sum(s)/size(X,1));

1405766.85128888
\end{verbatim}

  \newpage

\item[g.] Plot 2D point cloud.

  In Figure 6 below, the plot of the 2D point cloud is shown.
  \begin{figure}[H]
    \centering \fbox{
      \begin{minipage}{.90\linewidth}
        \includegraphics[width=\linewidth]{code/q2g.png}
        \caption{In MATLAB, the 400 digits after projection by PCA is shown with
          red dots indicating three's and blue dots indicating eight's. The axes
          are the first and second principle components from projection.}
        \label{fig:q2a1}
      \end{minipage}}
  \end{figure}

\item[3.] Naive Bayes
\item[a.] Estimating \( \pi \) probabilities.
\item[b.] Class conditional distribution for English.
\item[c.] Make bag-of-words.
\item[d.] Computing \( \hat{p}(x | y) \).
\item[e.] Posterior \( \hat{p}(x | y) \).
\item[f.] Evaluating performance of classifier.
\item[g.] Limiting training sample.

% \item[a.] 95\% confidence interval.

%   Below, I provide the code for constructing the 95\% confidence interval.
% \begin{minted}{python}
% from scipy.stats import norm, ttest_rel, t
% import numpy as np
% # 2a
% f = open('./data.txt', 'r')
% A = []
% B = []
% for line in f:
%   line = line.strip().split()
% A.append(float(line[0]))
% B.append(float(line[1]))

% a = np.array(A)
% b = np.array(B)

% mean_a = np.mean(a)
% mean_b = np.mean(b)

% std_a = np.std(a,ddof=1)
% std_b = np.std(b,ddof=1)

% n = len(a)
% print(n)
% alpha = 0.05
% tci_a = t.interval(1-alpha, n-1, mean_a, std_a/np.sqrt(n))
% tci_b = t.interval(1-alpha, n-1, mean_b, std_b/np.sqrt(n))

% print('Algorithm A CI:')
% print(tci_a)
% print()
% print('Algorithm B CI:')
% print(tci_b)
% \end{minted}

% The output of the Python script is reported.

% Algorithm A CI:
% (np.float64(2.3434452353035717), np.float64(3.8232214313630952))

% Algorithm B CI:
% (np.float64(1.5268346967576174), np.float64(2.9731653032423826))

% Based on the reporting, we can say that two confidence intervals overlap.
% However, we cannot say that one algorithm is better than the other since they do
% indeed overlap, and a paired t-test is required to determine if one is better
% than the other in a statistically meaningful way.
% \item[b.] Two-sided paired t-test for null-hypothesis.

%   The code for computing the t-test is as follows.
% \begin{minted}{python}
% print(ttest_rel(a, b).confidence_interval)
% \end{minted}
% This outputs the following result.

% TtestResult(statistic=np.float64(2.277867258047101),
% pvalue=np.float64(0.043699584804600185), df=np.int64(11))

% Since the standard t-distribution with 11 degrees of freedom and 95\% confidence
% is known to be 2.201, and so interval being [-2.201, 2.201], we know that our
% t-stat of 2.278 goes out of this interval, so we are safe to reject the null
% hypothesis and say that one algorithm has a statistically difference in
% performance compared to the other. From this experiment, we can see that
% two-sided paired t-test to find a statistically significant difference
% between two algorithms is concrete and deterministic while comparing two
% separate confidence intervals does not really allow us to conclude anything,
% even if they do overlap as they did in this case.

% \item[c.] Report p-value for testing done in b.

%   The p-value as reported in part b is 0.044. We know that the two algorithms
%   are statistically different because our p-value falls below our 0.05
%   significance threshold. This is another way to determine if the results of two
%   models are statistically different from each other.
% \item[d.] Describe pseudocode for bootstrapping method for this problem.

%   The pseudocode for bootstrapping is as follows.

%   Obtain F1 score for each fold.

%   Sort them in increasing order.

%   Compute the top .25 quantile.

%   Compute the bottom .25 quantile.

%   The CI is [F1 - top quantile, F1 + lower quantile]

% \item[e.] Implement and perform bootstrapping paired test with computed p-value.

%   \newpage
% \item[3.] Linear regression.

% \item[a.] Plot year versus ice days.

%   Below, I provide the plot of annual ice cover days for Lake Mendota and Lake
%   Monona.

%   \begin{figure}[H]
%     \centering \fbox{
%       \begin{minipage}{.98\linewidth}
%         \includegraphics[width=\linewidth]{p2/ice_cover.png}
%         \caption{Plot of annual ice cover days for Lake Mendota (red) and Lake
%           Monona (green) using Python.}
%         \label{fig:3a}
%       \end{minipage}}
%   \end{figure}

%   Below, I provide the plot of the difference in annual days of ice cover
%   between Lake Monona and Lake Mendota (Lake Monona - Lake Mendota).

%   \begin{figure}[H]
%     \centering \fbox{
%       \begin{minipage}{.98\linewidth}
%         \includegraphics[width=\linewidth]{p2/diff.png}
%         \caption{Plot of the difference in annual days of ice cover between Lake
%         Monona and Lake Mendota using Python.}
%         \label{fig:3b}
%       \end{minipage}}
%   \end{figure}

%   Below is the Python script used to generate the two plots.
% \begin{minted}{python}
% import pandas as pd
% import matplotlib.pyplot as plt
% import numpy as np

% # load in csv data
% col = 'Days of Ice Cover'

% mendota_df =
% pd.read_csv('mendota.csv')
% .loc[5:175].dropna(subset=[col]).iloc[::-1].reset_index(drop=True)

% monona_df =
% pd.read_csv('monona.csv')
% .loc[6:176].dropna(subset=[col]).iloc[::-1].reset_index(drop=True)

% # 3a
% plt.figure(1)
% plt.plot([x for x in range(1855,2019)],mendota_df[col], color='red')
% plt.plot([x for x in range(1855,2019)],monona_df[col], color='green')
% plt.xlabel('Year')
% plt.ylabel(col)
% plt.title('Annual Days of Ice Cover for Two Lakes')
% plt.legend(['Mendota', 'Monona'])
% plt.savefig('ice_cover.png')
% plt.show()

% diff = []
% for i,j in zip(monona_df[col],mendota_df[col]):
%     diff.append(i-j)

% plt.figure(2)
% plt.plot([x for x in range(1855,2019)],diff, color='blue')
% plt.xlabel('Year')
% plt.ylabel('Ice Days_Monona - Ice Days_Mendota')
% plt.title('Difference in Annual Days of Ice Cover between Two Lakes')
% plt.savefig('diff.png')
% plt.show()
% \end{minted}

% \item[b.] Split the datasets into training and testing. Compute std and mean for
%   the two lakes respetively.

%   The results are as follows.

%   Mendota Mean:
%   107.19

%   Mendota STD:
%   16.74

%   Monona Mean:
%   108.48

%   Monona STD:
%   18.12

%   Below is the Python script used to compute the means and STDs.
% \begin{minted}{python}
% # 3b
% split = mendota_df.index[mendota_df['Winter'] == '1970-71'].tolist()[0] + 1
% mendota_df_train = mendota_df.iloc[:split]
% mendota_df_test = mendota_df.iloc[split:]
% split = monona_df.index[monona_df['Winter'] == '1970-71'].tolist()[0] + 1
% monona_df_train = monona_df.iloc[:split]
% monona_df_test = monona_df.iloc[split:]

% mendota_a = np.array(mendota_df_train[col])
% monona_a = np.array(monona_df_train[col])

% print('Mendota Mean:')
% print(np.mean(mendota_a))
% print('Mendota STD:')
% print(np.std(mendota_a,ddof=1))
% print()
% print('Monona Mean:')
% print(np.mean(monona_a))
% print('Monona STD:')
% print(np.std(monona_a,ddof=1))
% \end{minted}

% \item[c.] Using training sets, train a linear regression model.

%   Scikit-learn in Python was used for this section.

%   Below I provide the Python script for training a linear regression model.
% \begin{minted}{python}
% # 3c
% from sklearn.linear_model import LinearRegression
% from sklearn.model_selection import train_test_split
% from sklearn.metrics import mean_squared_error

% lr = LinearRegression()
% monona_df_train['ones'] = 1
% monona_df_test['ones'] = 1
% lr.fit(monona_df_train[['ones','Winter',col]], mendota_df_train[[col]])
% mendota_pred = lr.predict(monona_df_test[['ones','Winter', col]])

% weights = lr.coef_
% print(weights)
% intercept = lr.intercept_
% print(intercept)
% \end{minted}

% The results are as follows:

% Feature weights: [[0., 0.04122457, 0.85295064]]

% Intercept:[-64.1827663]

% These are the relevant outputs of the Scikit-learn LinearRegression model. One
% thing to note is that although we are trying to get an intercept in our OLS
% solution by modeling our features as (1, x, $y_{monona}$), the LinerRegression
% model in Scikit-learn does not seem to care about an artificial `1' appended to
% the feature vector and will in fact produce the same result without the `1'
% column, so we can get away with just (x, $y_{monona}$). As reported above, the
% feature weights which represent $(B_0, B_1,B_2)$ omits the $B_0$ value; probably
% because it ignores a column of 1's like I mentioned. Instead, it will report our
% $B_0$ (which is our intercept) in its own category called `intercept'.

% \item[d.] Mean squared error on the test set.

%   Below, I provide the Python script to compute the MSE of our predicted values
%   from our Scikit-learn LinearRegression model. The computation of MSE is also
%   provided in the Scikit-learn library.
% \begin{minted}{python}
% # 3d
% mse = mean_squared_error(mendota_df_test[[col]], mendota_pred)
% print(mse)
% \end{minted}

% The result of this script is reported below.

% Mean sqaure error: 124.26409483990123

% This is a fairly nice MSE since it is the square of the mean error value. It
% means our linear regression model is on average about 11 days off the mark every
% year.

% \item[e.] Train a linear regression model using Monona.

%   Below, I provide the Python script for retraining our linear regression model
%   on our new feature vector which just omits $y_{monona}$.
% \begin{minted}{python}
% # 3e

% lr2 = LinearRegression()
% lr2.fit(monona_df_train[['ones', 'Winter']],mendota_df_train[[col]])
% mendota_pred2 = lr2.predict(monona_df_test[['ones', 'Winter']])
% mse2 = mean_squared_error(mendota_df_test[[col]], mendota_pred2)

% weights2 = lr2.coef_
% intercept2 = lr2.intercept_

% print('Feature weights: ' + str(weights2))
% print('Intercept:' + str(intercept2))
% print('Mean sqaure error: ' + str(mse2))
% \end{minted}

% I report the newly trained linear model below.

% Feature weights: [[ 0., -0.15629877]]

% Intercept:[406.11105985]

% Mean sqaure error: 418.14436543672855

% As seen above, the MSE is much higher when the linear model only has the year
% and cannot reference $y_{monona}$ in order to predict $y_{mendota}$. This
% intuitively makes sense. Remember that the first element of the feature weights
% which should be intercept is simply reported in a separate variable called
% `intercept' in Scikit-learn's LinerRegression model.

%   \begin{itemize}
%   \item[i.] Interpret the sign of $\gamma_1$.

%     $\gamma_1$ can be interpreted as the slope of our x variable. Since it is
%     negative, we can say that it derived a directly negative correlation between
%     the year and the number of ice days. In other words, as the years go by and
%     `increases', the number of ices days decrease. Even more simply put, our
%     model predict a warming trend.
%   \item[ii.] Assessing a viewpoint formed from the model.

%     The stated viewpoint could be wrong because looking at our model again, it
%     can be seen that the strongest weight $B_2=0.85$ is much greater than the
%     $B_1=0.04$ that the analysts are referring to. $B_2$ is the weigh of feature
%     $y_{monona}$ on label $y_{mendota}$ and since they have a strong correlation
%     to each other and are trending downwards together, there is little weight on
%     $B_1$ which is the number of years since feature $y_{monona}$ is doing all
%     the `heavy lifting'. In fact, it is more accurate to say that $B_1$ is
%     zero-ish weight, and it was a coincidence that it happened to be barely
%     positive as a part of the OLS solution.
%   \end{itemize}

% \item[4.] Maximum likelihood estimation for linear regression.

% \item[a.] Likelihood of the ith data point.

%   $Laplace(\mu ,b) \rightarrow f(z|\mu , b) = \frac{1}{2b}e^{-\frac{|z-u|}{b}}$

%   $Laplace(w^Tx,1) \rightarrow f(z|w^Tx, 1) = \frac{1}{2}e^{-|z-u|}$

%   $P(y_i|x_iw)=f(y_i|w^Tx_i,1)$

%   $= \frac{1}{2}e^{-|y_i-w^Tx_i|}$
% \item[b.] Log-likelihood of the dataset.

%   $L = P(y_1...,y_n|x_1,...,x_n,w)$

%   $=\prod_{i=1}^{n}P(y_i|x_i,w)$

%   $=\prod_{i=1}^{n}(\frac{1}{2}e^{-|y_i-w^Tx_i|})$

%   $\log(L) = \log(\prod_{i=1}^{n}(\frac{1}{2}e^{-|y_i-w^Tx_i|}))$

%   $=\sum_{i=1}^{n}\log(\frac{1}{2}e^{-|y_i-w^Tx_i|})$

%   $=n\log(\frac{1}{2})+\sum_{i=1}^{n}\log(e^{-|y_i-w^Tx_i|})$

%   $=n\log(\frac{1}{2})+\sum_{i=1}^{n}(-|y_i-w^Tx_i|)$

%   $=n\log(\frac{1}{2})-\sum_{i=1}^{n}|y_i-w^Tx_i|$

%   $w^{MLE}=\argmax_w\sum_{i=1}^{n}(\log(\frac{1}{2}) - |y_i-w^Tx_i|)$

%   $w^{MLE}=\argmin_w\sum_{i=1}^{n}|y_i-w^Tx_i|$

%   From our derivation, the optimization problem is to find the value of $w$ that
%   would minimize $\sum_{i=1}^{n}|y_i-w^Tx_i|$.

%   This is not equivalent to the least squares problem because our derivation
%   above shows we are minimizing the residuals of the points to a line which is
%   consistent with our study of linear regression. In least squares however, you
%   want to take the zero derivative of a convex, quadratic function in order to
%   find the global minimum, and the derived $w^{MLE}$ from finding this is, in
%   general, not the same problem as minimizing residuals.

% \item[c.] Maximum likelihood estimator $w$ under the above model.

%   $w^{MLE}=\argmin_w\sum_{i=1}^{n}|y_i-w^Tx_i|$

%   $w^{MLE}=\min_w(|3-w(1)|+|5-w(1)|+|6-w(1)|)$

%   $w^{MLE}=\min_w(|3-w|+|5-w|+|6-w|)$

%   Then, let $f(w) = |3-w|+|5-w|+|6-w|$. Below, I provide the plot of $f(w)$.

%   % \begin{figure}[H]
%   %   \centering \fbox{
%   %   \begin{minipage}{.90\linewidth}
%   %     \includegraphics[width=\linewidth]{Screenshot 2024-10-03 at 1.00.00 PM.png}
%   %     \caption{Plot of $f(w)$ in MATLAB. The clicked point at (5,3) is the
%   %       global minimum of $f(w)$.}
%   %     \label{fig:4c}
%   %   \end{minipage}}
%   % \end{figure}

%   From the plot above, we can see that the function $f(w)$ is minimized when
%   $w=5$, so we have $w^{MLE}=\min_w(|3-w|+|5-w|+|6-w|) = 5$.
% \item[B.] Making a world coordinate system.
%   Below in Figure 4, I provide the world coordinate system for the image provided
%   for hw03.

%   \begin{figure}[H]
%     \centering \fbox{
%       \begin{minipage}{.98\linewidth}
%         \includegraphics[width=\linewidth]{IMG_0862.jpeg}
%         \caption{World coordinate system for provided image for hw03. The X, Y,
%           and Z axes are clearly labeled, and the direction of the axes' arrows
%           indicate in which direction they grow. 15 blue points have been
%           selected for this experiement as shown, and their (X,Y,Z) coordinates
%           are labeled in yellow.}
%         \label{fig:b1}
%       \end{minipage}}
%   \end{figure}

%   These 15 blue world coordinate points in Figure 4 will be used for Part C to
%   compare how two cameras map these 3D world coordinates to 2D. The files
%   world\_coords.txt and image\_coords.txt have also been provided in this
%   submission for Part C. The world coordinates of the points are as they are in
%   Figure 4, and the image coordinates (X, Y) were obtained using the
%   `datacursormode on' feature in MATLAB and manually clicking on the labeled
%   blue points.

% \item[C.] Analyzing how a camera matrix maps points from 3D to 2D.

%   Below in Figure 5, I provide the visual representation of our clicked world
%   coordinates along with the two camera estimates of our world coordinates.

%   \begin{figure}[H]
%     \centering \fbox{
%       \begin{minipage}{.98\linewidth}
%         \includegraphics[width=\linewidth]{Screenshot 2024-09-29 at 11.19.16 PM.png}
%         \caption{The red clicked points are our world coordinates from Figure 4
%           moved onto this black canvas to compare to our first (green) and
%           second (blue) camera estimates. Produced in MATLAB.}
%         \label{fig:c1}
%       \end{minipage}}
%   \end{figure}

%   In Figure 5 above, we can see that both the first and camera estimates are
%   visually close to our red clicked points. One thing to note when plotting the
%   estimates was to flip the X and Y coordinates generated from the given camera
%   matrices which we explored the reason for in Part A.

%   The RMSE of the distance between the camera estaimte point and the actual
%   clicked point was calculated for both cameras in MATLAB. Again, when
%   calculating the distsances between points, we have to flip the coordinates
%   generated from the camera matrices, or flip the image coordinates of the world
%   coordinates. The key thing is to be consistent with our coordinate system
%   (either work in X vs Y values or work with indices of a matrix). The RMSEs are
%   as follows:

%   $RMSE_{camera1}=23.9$

%   $RMSE_{camera2}=40.7$

%   The RMSE of the camera estimated points and actual world coordinate points is
%   also known as the re-projection error as stated in the homework. From the
%   reported RMSEs of the cameras, the first camera is better quantitatively
%   speaking, and from scrutinizing Figure 5, one can see that the first camera
%   (green) points are closer to the clicked (red) points than the second (blue)
%   camera points on average.

%   Thus, the first camera's matrix is more accurate.

\end{enumerate}

\end{document}
